\documentclass{article}
\title{Exercise 7}
\begin{document}
\maketitle
\section{A note on Maths in Prolog}

Maths in prolog is weird, simple assignment of mathematics doesn't work because
arithmetic is considered a separate data type. In order to assign the result of
an equation you have to use the keyword \verb|is|

\section{Ex 1}

Make a simple calculator which prints out the result of a formula (Hint: use
\verb|write/1|)

\verb|calc(X)|

\section{Ex 2}

Make a \verb|split| relationship which is true when a given list is split into 2
sublists

\verb|split/3|

\section{Ex 3}

Make a predicate which holds when all the elements of a list are positive (i.e.
\textbf{greater} than 0)

\verb|positives/1|

\end{document}
